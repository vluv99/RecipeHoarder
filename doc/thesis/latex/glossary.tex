\newglossaryentry{TypeScript}
{
        name=TypeScript,
        description={Egy JavaScript-re forduló program nyelv. A JavaScript Microsoft álltal tovább fejlesztett verziója.}
}

\newglossaryentry{Framework}
{
        name=Framework,
        description={Keretrendszer}
}

\newglossaryentry{AngularJS}
{
        name=AngularJS,
        description={Az Angular keretrendszer előző verziója. A jelenlegi Angulár ennek teljes átírása.}
}

\newglossaryentry{SinglePageApp}
{
        name=Single Page Alkalmazás,
        description={Egy olyan webes alkalmazás, ahol egy index.html lap van újra és újra feltöltve dinamikusan adatokkal.}
}

\newglossaryentry{Metaadatok}
{
        name=Meta adatok,
        description={Egy objektumhoz tartozó annak  részletesebb adatair tároló adatösszegség.}
}

\newglossaryentry{serviceWorker}
{
        name=Service Worker,
        description={Egy szkript, amit a böngésző a háttérben futtat. 
        Ezt be lehet állítani, hogy különböző folyamatokat végezzen, úgy hogy a felhasználónak nem kell megnyitva hagynia a weboldalt.}
}

\newglossaryentry{cache}
{
        name=cache,
        description={A böngésző által ideiglenesen eltárolt adatok amiket egy beállítható ideig nem fog újból lekérdezni a szervertől.}
}

\newglossaryentry{Apache}
{
        name=Apache,
        description={Egy elterjedt webszerver.}
}

\newglossaryentry{NGINX}
{
        name=NGINX,
        description={Az \Gls{Apache}-val együtt az egyik legelterjedtebb webszerver. Sok esetben revers proxinak is szokás használni.}
}

\newglossaryentry{HTTP}
{
        name=HTTP,
        description={Hypertext Transfer Protocol, egy applikáció szinten működő rendszer, ami webböngészők és webes szerverek közötti komminikációt tesz lehetővé.}
}

\newglossaryentry{dependencyInjection}
{
        name=Dependency Injection,
        description={Egy olyan technika, ahol egy objektum konstruktorán keresztül kap egy másik objektumot, amitől függ.}
}

\newglossaryentry{JAVA}
{
        name=JAVA,
        description={Egy magas szintű, objektumorientált programozási nyelv.}
}

\newglossaryentry{cloudFunctions}
{
        name=Cloud Functions,
        description={A Firebase szerver oldali funkcióinak az összesítő neve. A fejlesző csak a futtatni kíván}
}
