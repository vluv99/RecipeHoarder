\newglossaryentry{TypeScript}
{
        name=TypeScript,
        description={JavaScript-re forduló program nyelv. A JavaScript Microsoft álltal tovább fejlesztett verziója}
}

\newglossaryentry{Framework}
{
        name=Framework,
        description={Keretrendszer}
}

\newglossaryentry{AngularJS}
{
        name=AngularJS,
        description={Az Angular keretrendszer előző verziója. A jelenlegi Angular ennek teljes átírása}
}

\newglossaryentry{SinglePageApp}
{
        name=Single Page Alkalmazás,
        description={Olyan webes alkalmazás, ahol egy index.html lap van újra és újra feltöltve dinamikusan adatokkal}
}

\newglossaryentry{Metaadatok}
{
        name=Metaadatok,
        description={Egy adatot leíró adat, például annak utolsó módosítási ideje}
}

\newglossaryentry{serviceWorker}
{
        name=Service Worker,
        description={Egy szkript, amit a böngésző a háttérben futtat. 
        Ezt be lehet állítani, hogy különböző folyamatokat végezzen, úgy hogy a felhasználónak nem kell megnyitva hagynia a weboldalt}
}

\newglossaryentry{cache}
{
        name=cache,
        description={A böngésző által ideiglenesen eltárolt adatok, amiket egy beállítható ideig nem fog újból lekérdezni a szervertől}
}

\newglossaryentry{Apache}
{
        name=Apache,
        description={Egy elterjedt webszerver szoftver, ami az\Gls{NGINX} mellet a másik legnagyobb webszerver platform}
}

\newglossaryentry{NGINX}
{
        name=NGINX,
        description={Az \Gls{Apache}-val együtt az egyik legelterjedtebb webszerver. Sok esetben fordított proxy-nak is szokás használni}
}

\newglossaryentry{HTTP}
{
        name=HTTP,
        description={Hypertext Transfer Protocol, egy applikáció szinten működő rendszer, ami webböngészők és webes szerverek közötti komminikációt tesz lehetővé}
}

\newglossaryentry{dependencyInjection}
{
        name=Dependency Injection,
        description={Egy olyan technika, ahol egy objektum konstruktorán keresztül kap egy másik objektumot, amitől függ}
}

\newglossaryentry{JAVA}
{
        name=JAVA,
        description={Egy magas szintű, objektumorientált programozási nyelv}
}
 
\newglossaryentry{cloudFunction}
{
        name=Cloud Function,
        description={A Firebase szerver oldali funkcióinak az összesítő neve. A fejlesző csak a futtatni kívánt kódot kell fletöltenie, a platform karbantartása a Firebase feladata}
}

\newglossaryentry{JSON}
{
        name=JSON,
        description={/ JavaScript Object Notation. A Javasicript álltal használt objektum mentési formátum}
}

\newglossaryentry{regex}
{
        name=RegEx,
        description={Regular Expresion (Reguláris Kifejezés) egy szöveg értelmezésre használható minta felismerő nyelv}
}

\newglossaryentry{unicodeVulgarFraction}
{
        name=Unicode Vulgar Fraction,
        description={Unikódban meghatározott azon karakterek, amelyek egy törtet írnak le egyetlen karaterrel. pl.: ¼}
}


\newglossaryentry{API}
{
        name=API,
        description={Application Progamming Interface, különböző programok közötti kapcsolatot ír le. 
        Például: milyen URL címeket lehet meghívni egy szerveren adat lekérdezés céljából}
}

\newglossaryentry{USDA}
{
        name=USDA,
        description={Unister States Department of Agricuture, Amerikai Egyesült Államok Mezőgazdasági Minisztériuma}
}
